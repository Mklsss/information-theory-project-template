\documentclass[12pt,a4paper]{article}

% ------------------- Packages -------------------
\usepackage[margin=1in]{geometry}
\usepackage{setspace}
\usepackage{graphicx}
\usepackage{booktabs}
\usepackage{array}
\usepackage{amsmath,amssymb}
\usepackage{subcaption}
\usepackage{float}
\usepackage{hyperref}
\usepackage{fancyhdr}

\usepackage{CJKutf8}
\usepackage{indentfirst}

% Fonts: Times New Roman for text & math
\usepackage{newtxtext,newtxmath}
\usepackage{mdframed}

% --- 表格相关宏包 ---
\usepackage[table]{xcolor} % 表格颜色
\usepackage{textcomp}      % \textvisiblespace
\usepackage{arydshln}      % 虚线
\usepackage{tabularx}
\usepackage{makecell}
\usepackage{seqsplit}
\usepackage{colortbl}

% 统一定义颜色与列格式
\definecolor{tableheaderblue}{RGB}{230,240,255}
\newcommand{\code}[1]{\ttfamily\footnotesize\seqsplit{#1}}
\newcolumntype{S}{>{\centering\arraybackslash\ttfamily}p{0.06\textwidth}}
\newcolumntype{Z}{>{\raggedright\arraybackslash}p{0.26\textwidth}}

% ---- boxed excerpt with caption (可选) ----
\usepackage{tcolorbox}

% ------------------- Page Style -------------------
\onehalfspacing
\pagestyle{fancy}
\fancyhf{}
\lhead{Information Theory Project 1}  % 课程标题(可修改)
\rhead{\thepage}

% ------------------- Bibliography -------------------
\usepackage[backend=biber,style=ieee]{biblatex}
\addbibresource{ref.bib} % 参考文献库文件名(可修改)

% =====================================================
%                     DOCUMENT
% =====================================================
\setlength{\parindent}{2em}

\begin{document}

% ==================== COVER PAGE =====================
\begin{titlepage}
\centering
\thispagestyle{empty}

% ----- Logos (自行替换图片路径或删除) -----
\begin{figure}[H]
    \centering
    \includegraphics[width=14cm]{figures/logo.png}
\end{figure}

\vspace{1cm}

{\Large \textbf{University Name}}\\[6pt]
{\large \textbf{Course Name}}\\[6pt]
{\large Academic Year}\\[1.4cm]

{\Large \textbf{Coding Project:}}\\[6pt]
{\Large \textbf{Project Title Here}}\\[2cm]

% --------- Student info table ---------
\renewcommand{\arraystretch}{1.8}
\begin{center}
\newcolumntype{C}[1]{>{\centering\arraybackslash}p{#1}}
\begin{tabular}{|C{3.5cm}|C{3.5cm}|C{3.5cm}|C{3.5cm}|}
\hline
\textbf{Student's Chinese Name} &
\textbf{Student's English Name} &
\textbf{Student's UESTC ID} &
\textbf{Student's UoG ID} \\
\hline
% 填写学生信息行
% \begin{CJK}{UTF8}{gbsn}张三\end{CJK} & Zhang San & 2023xxxxx & 29xxxxx \\
% \hline
\end{tabular}
\end{center}

\vfill
\end{titlepage}

% =================== TOC ===================
\tableofcontents
\newpage

% =====================================================
%                    MAIN CONTENT
% =====================================================

\section{Introduction}
\subsection{Problem Background}
% 在此撰写问题背景

\subsection{Restatement of the Problem}
% 在此重新表述题目与任务

% =====================================================
\section{Methodology}
% 概述整体方法框架

\subsection{Baseline Huffman Coding}
% 描述基线方法

\subsubsection{Standard Huffman Coding}
% 标准 Huffman 方案

\subsubsection{Extended Huffman Coding}
% 扩展字符集的 Huffman 方案

\subsection{Lexical-aware Huffman}
% 词汇感知 Huffman 总体思路

\subsubsection{Dual-Alphabet Lexical-aware Huffman}
% 双字母表版本

\subsubsection{Tagged Lexical-aware Huffman}
% 大小写标签版本

\subsection{Improvements Based on Lexical-aware Huffman}
% 基于词汇模型的改进

\subsubsection{Hybrid Huffman Coding}
% Hybrid 机制

\subsubsection{Canonical Huffman Coding}
% Canonical 机制

% =====================================================
\section{Results and Discussions}
\subsection{Standard and Extended Huffman Coding}
% 放入标准 / 扩展实验结果与图表

\subsection{Lexical-aware Huffman Coding}
% 放入词汇模型实验结果

\subsection{Improvements}
% 放入 Hybrid / Canonical 对比结果与讨论

% =====================================================
\section{Project Evaluation and Further Discussion}
\subsection{Strengths}
% 总结方案优点

\subsection{Weaknesses}
% 总结方案不足

\subsection{Further Discussion}
% 可以写扩展思路和未来工作

% =====================================================
\section{Conclusion}
% 总结全文与主要结论

% =====================================================
\printbibliography

% =====================================================
\appendix
\section{Appendix: Source Code Description}
% 在附录中简要介绍各个代码文件与功能

\end{document}
